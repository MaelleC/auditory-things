%%%%% this line is 80 chars wide, please don't make longer lines %%%%%%%%%%%%%%%
The neuronal representation of sound is the result of the encoding of acoustic 
signals by the peripheral auditory system. 
The spike trains resulting from this encoding are influenced, 
among other factors, by the refractory period of the auditory nerve fibers. 
In fact, for example we learn in \cite{AvissarPapier} that, for the encoding of pure tones, 
the spike timing precision depends on the ratio of  
the refractory period to the stimulus period, 
and the entrainment of nerve responses to the stimulus was better with the refractory period. 
We also know from \cite{BerryMeister} that the refractoriness of neurons may make their signals 
more reliable. 


This project aims to study the effects of  
refractoriness on sound encoding in the peripheral 
auditory system.  
First, we studied this effect based on an ad-hoc measure of response spike trains,  
the rate-modulation depth, for four kinds of stimuli.
In \cite{Deger}, point processes with refractoriness were studied and mathematical 
predictions were made for the Fourier coefficients of their response.
The second part of the project consisted on trying to see 
if the results on the model matches these predictions 
when the stimulus is a modulated pure tone.

%we have better entrainment of spike trains and often also better precision 
%in time encoding. 
%%%

%This project studied the effects of this neural property 
%on the resulting encoded spike trains. 

For this aim, a model of the peripheral auditory system was used
\cite{Model1, Model2, Model3}, in which the absolute refractory period has been be modified. 
Virtual experiments were run on the two versions of the model and the resulting 
spike trains were compared to see the influence of the refractory period.
 
Now, before going any deeper into the model, we should introduce the basic 
physiology of the peripheral auditory system.

%On website

%Acoustic signals are encoded as spike trains by auditory nerve fibers. The time-dependent firing rate and other aspects of spike train statistics depend, among other things, on the refractory period of the nerve fibers. This project aims to understand the influence of the refractory period on the neuronal representation of sounds.

%A phenomenological model of the peripheral auditory system [1,2] will be used to perform virtual experiments. The model allows to modify the refractory period of the nerve fibers. The responses of the model to a selection of auditory stimuli will be recorded and characterized as a function of the neuronal refractory period. If time permits, specific theoretical predictions like frequency doubling [3] will be tested.
