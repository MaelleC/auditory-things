%%%%% this line is 80 chars wide, please don't make longer lines %%%%%%%%%%%%%%%
The neuronal representation of sound is the result of the encoding of acoustic 
signals done through the auditory system. The spike trains resulting from this 
encoding are influenced, among other factors, by the refractory period of the 
auditory nerve fibers. This project studied the effects of this neural property 
on the resulting encoded spike trains.

For this aim, it has used a model of the auditory system
\cite{Model1, Model2, Model3} , in which the refractory period can be modified. 
Virtual experiments were run on the two versions of the model and the resulting 
spike trains were compared to see the influence of the refractory period. 
Before going any deeper on the model, we should remind us some things 
about the auditory system.

%Acoustic signals are encoded as spike trains by auditory nerve fibers. The time-dependent firing rate and other aspects of spike train statistics depend, among other things, on the refractory period of the nerve fibers. This project aims to understand the influence of the refractory period on the neuronal representation of sounds.

%A phenomenological model of the peripheral auditory system [1,2] will be used to perform virtual experiments. The model allows to modify the refractory period of the nerve fibers. The responses of the model to a selection of auditory stimuli will be recorded and characterized as a function of the neuronal refractory period. If time permits, specific theoretical predictions like frequency doubling [3] will be tested.
